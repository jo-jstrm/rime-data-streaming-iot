\documentclass{standalone}
\usepackage[dvipsnames, fixpdftex]{xcolor}
\usepackage{forest}

\usepackage{pgfplots}
\pgfplotsset{compat=1.15}

% Code from Christian Feuersänger
% https://tex.stackexchange.com/questions/54794/using-a-pgfplots-style-legend-in-a-plain-old-tikzpicture#54834

% argument #1: any options
\newenvironment{customlegend}[1][]{%
    \begingroup
    % inits/clears the lists (which might be populated from previous
    % axes):
    \csname pgfplots@init@cleared@structures\endcsname
    \pgfplotsset{#1}%
}{%
    % draws the legend:
    \csname pgfplots@createlegend\endcsname
    \endgroup
}%

% makes \addlegendimage available (typically only available within an
% axis environment):
\def\addlegendimage{\csname pgfplots@addlegendimage\endcsname}

%%--------------------------------

% definition to insert numbers
\pgfkeys{/pgfplots/number in legend/.style={%
        /pgfplots/legend image code/.code={%
            \node at (0.295,-0.0225){#1};
        },%
    },
    /pgfplots/shape in legend/.style 2 args={%
    /pgfplots/legend image code/.code={%
        \node[draw,#1,fill=#2,minimum width=3mm] at (0.295,-0.0225){};
        },%
    },
}

\begin{document}

\begin{forest}
for tree={circle,draw,minimum size = 2.25em, l sep=0.4em, s sep=7.5em, outer sep=0.1em}
[0, name=zero, edge=white  
    [1, name=one, edge=white  
      [3, name=three, shape=star, edge=white]
      [4, name=four, edge=white]
    ]
    [2, name=two, shape=star, edge=white, for tree={l sep=0.7em}
      [5, name=five, edge=white ] 
      [6, name=six, edge=white] 
    ]  
] 
\draw[->, color=black] (five) to node[pos=0.5,xshift=-3mm,yshift=1.5mm]{1.)} (two);
\draw[->, color=black] (two) to node[pos=0.5,xshift=2.0mm,yshift=1.5mm]{2.)} (six);
\draw[->, color=black] (two) to node[pos=0.5,xshift=2.0mm,yshift=1.5mm]{2.)} (zero);
\draw[->, color=black] (zero) to node[pos=0.5,xshift=-2.0mm,yshift=2.0mm]{3.)} (one);
\draw[->, color=black] (one) to node[pos=0.5,xshift=-3.0mm,yshift=1.5mm]{4.)} (three);
\draw[->, color=black] (one) to node[pos=0.5,xshift=2.0mm,yshift=1.5mm]{4.)} (four);
\end{forest}

\end{document}