\documentclass[border=9]{standalone}
\usepackage[dvipsnames, fixpdftex]{xcolor}
\usepackage{tikz}

\usetikzlibrary{trees,decorations,shapes.geometric,backgrounds}

\usepackage{pgfplots}
\pgfplotsset{compat=1.15}

% Circle a character
\newcommand*\circled[1]{\tikz[baseline=(char.base)]{
            \node[shape=circle,draw,inner sep=0.5pt,fill=white] (char) {#1};}}

% change color of node after declaration
\newcommand\labeltextof[1]{\csname labeltextof@#1\endcsname}
\newcommand{\aftercolorof}[2]{% #1 is the color, #2 us the node
  \path (#2.center) node[#1] (#2-2) {\labeltextof{#2}};
}

% Legend
% Code from Christian Feuersänger
% https://tex.stackexchange.com/questions/54794/using-a-pgfplots-style-legend-in-a-plain-old-tikzpicture#54834

% argument #1: any options
\newenvironment{customlegend}[1][]{%
    \begingroup
    % inits/clears the lists (which might be populated from previous
    % axes):
    \csname pgfplots@init@cleared@structures\endcsname
    \pgfplotsset{#1}%
}{%
    % draws the legend:
    \csname pgfplots@createlegend\endcsname
    \endgroup
}%

% makes \addlegendimage available (typically only available within an
% axis environment):
\def\addlegendimage{\csname pgfplots@addlegendimage\endcsname}

%%--------------------------------

% definition to insert numbers
\pgfkeys{/pgfplots/number in legend/.style={%
        /pgfplots/legend image code/.code={%
            \node at (0.295,-0.0225){#1};
        },%
    },
    /pgfplots/shape in legend/.style 2 args={%
    /pgfplots/legend image code/.code={%
        \node[draw,#1,fill=#2,minimum width=3mm] at (0.295,-0.0225){};
        },%
    },
}
% End Legend

\makeatletter

\newcount\tikzcountchildi
\newcount\tikzcountchildii
\newcount\tikzcountchildiii
\newcount\tikzcountchildiv
\newcount\tikzcountchildv
\newcount\tikzcountchildvi

\tikzset{
    garlic growth/.style={
        growth function=\tikz@grow@garlic,
        /tikz/mmap/name=#1,mmap/#1/.is family,
        mmap/initialize counts,
        execute at end scope={%
            \tikz@mmap@store@aux{n-i}{\the\tikzcountchildi}%
            \tikz@mmap@store@aux{n-ii}{\the\tikzcountchildii}%
            \tikz@mmap@store@aux{n-iii}{\the\tikzcountchildiii}%
            \tikz@mmap@store@aux{n-iv}{\the\tikzcountchildiv}%
            \tikz@mmap@store@aux{n-v}{\the\tikzcountchildv}%
            \tikz@mmap@store@aux{n-vi}{\the\tikzcountchildvi}
        }
    },
    mmap/.cd,initialize counts/.code={
        \tikzcountchildi=0%
        \tikzcountchildii=0%
        \tikzcountchildiii=0%
        \tikzcountchildiv=0%
        \tikzcountchildv=0%
        \tikzcountchildvi=0%
    }
}

\def\tikz@mmap@store@aux#1#2{%
    \immediate\write\@mainaux{\string\expandafter\xdef\noexpand\csname pgfk@/tikz/mmap/\pgfkeysvalueof{/tikz/mmap/name}/#1\string\endcsname{#2}}
}%

\def\tikz@mmap@get@from@aux#1#2{%
    \ifcsname pgfk@/tikz/mmap/\pgfkeysvalueof{/tikz/mmap/name}/#1\endcsname
        \edef#2{\csname pgfk@/tikz/mmap/\pgfkeysvalueof{/tikz/mmap/name}/#1\endcsname}%
    \else
        \edef#2{0}%
    \fi
}

\newcount\tikzcountgrandchild

\def\tikz@grow@garlic{%
  \pgfmathsetmacro{\pgfutil@tempa}{\pgfkeysvalueof{/tikz/mmap/overall rotation}}%
  \ifcase\tikztreelevel 
  \or
   \global\advance\tikzcountchildi by1\relax%
   \tikz@mmap@get@from@aux{n-i}{\myni}%
   \tikz@mmap@get@from@aux{n-ii}{\mynii}%
   \tikz@mmap@get@from@aux{n-1-\the\tikzcountchildi}{\mynall}%
   \tikz@mmap@get@from@aux{p-1-\the\tikzcountchildi}{\mypi}%
   \ifnum\mynii>0
    \pgfmathsetmacro{\pgfutil@tempa}{\pgfutil@tempa+\pgfkeysvalueof{/tikz/mmap/sign}%
        *(\pgfkeysvalueof{/tikz/mmap/child weight}*(\mypi)*360/\mynii+%
            (1-\pgfkeysvalueof{/tikz/mmap/child weight})*(\tikzcountchildi-1)*360/\myni)}%
   \fi
  \or
   \global\advance\tikzcountchildii by1\relax%
   \tikz@mmap@get@from@aux{n-ii}{\mynii}%
   \tikz@mmap@get@from@aux{n-1-1}{\mynall}%
   \ifnum\mynii>0
    \pgfmathsetmacro{\pgfutil@tempa}{\pgfutil@tempa+\pgfkeysvalueof{/tikz/mmap/sign}%
    *(\tikzcountchildii-1-\mynall/2)*360/\mynii}%
   \fi
   \ifnum\tikznumberofcurrentchild=1\relax
    \tikz@mmap@store@aux{n-1-\the\tikzcountchildi}{\the\tikznumberofchildren}%
    \tikz@mmap@store@aux{p-1-\the\tikzcountchildi}{\the\numexpr\tikzcountchildii-1}%
    \tikz@mmap@store@aux{a-1-\the\tikzcountchildi-\the\tikzcountchildii}{\pgfutil@tempa}%
   \fi   
  \or
   \global\advance\tikzcountchildiii by1\relax%
   \ifnum\tikznumberofcurrentchild=1\relax
    \tikz@mmap@store@aux{n-1-\the\tikzcountchildi-\the\tikzcountchildii}{\the\tikznumberofchildren}%
    \tikz@mmap@store@aux{p-1-\the\tikzcountchildi-\the\tikzcountchildii}{\the\numexpr\tikzcountchildiii-1}%
   \fi   
   \tikz@mmap@get@from@aux{n-iii}{\myniii}%
   \tikz@mmap@get@from@aux{a-1-1-1}{\bettera}%
   \tikz@mmap@get@from@aux{n-1-1-1}{\mynall}%
   \ifdim\bettera pt=0pt\relax
   \else
    \pgfmathsetmacro{\pgfutil@tempa}{\bettera}%
   \fi
   \ifnum\myniii>0
    \pgfmathsetmacro{\pgfutil@tempa}{\pgfutil@tempa+\pgfkeysvalueof{/tikz/mmap/sign}%
     *(\tikzcountchildiii-1-\mynall/2)*360/\myniii}%
   \fi  
  \or
   \global\advance\tikzcountchildiv by1\relax%
   \tikz@mmap@get@from@aux{n-iv}{\myniv}%
   \ifnum\myniv>0
    \pgfmathsetmacro{\pgfutil@tempa}{\pgfutil@tempa+\pgfkeysvalueof{/tikz/mmap/sign}*(\tikzcountchildiv-1)*360/\myniv}%
   \fi  
  \or
   \tikz@mmap@get@from@aux{n-v}{\mynv}%
   \pgfmathsetmacro{\pgfutil@tempa}{\pgfutil@tempa+\pgfkeysvalueof{/tikz/mmap/sign}*(\tikzcountchildv-1)*360/\mynv}%
   \ifnum\mynv>0
    \global\advance\tikzcountchildv by1\relax%
   \fi
  \or
   \global\advance\tikzcountchildvi by1\relax%
   \tikz@mmap@get@from@aux{n-vi}{\mynvi}%
   \ifnum\myvi>0
    \pgfmathsetmacro{\pgfutil@tempa}{\pgfutil@tempa+(\tikzcountchildvi-1)*360/\mynvi}%
   \fi  
  \fi
  \pgftransformreset% 
  \pgftransformshift{\pgfpoint{\pgfkeysvalueof{/tikz/mmap/overall xshift}}%
        {\pgfkeysvalueof{/tikz/mmap/overall yshift}}}%
  \pgftransformrotate{\pgfutil@tempa}%
  \pgftransformxshift{\the\tikzleveldistance}%
}

\makeatother

\begin{document}

\tikzset{mmap/.cd,
    name/.initial=undef,
    overall rotation/.initial=0,
    overall xshift/.initial=0pt,
    overall yshift/.initial=0pt,
    sign/.initial=1,
    child weight/.initial=0.5,
    /tikz/.cd,
    Xshift/.style={xshift=#1,mmap/overall xshift=#1},
    Yshift/.style={yshift=#1,mmap/overall yshift=#1},
    branch color/.style={
        concept color=#1!80,ball color=#1!50,
        every child/.append style={concept color=#1!50},
    }
}

\begin{tikzpicture}[garlic growth=A,mmap/child weight=0,
                    mmap/overall rotation=15,mmap/sign=-1,
                    shape = circle,
                    level 1/.style={shape = diamond, minimum size=1.75em,
                        level distance=1.2cm,
                    },
                    level 2/.style={shape = diamond, minimum size=1em,
                        level distance=2.4cm,
                    },
                    level 3/.style={shape = circle, minimum size=1mm,
                        level distance=3.6cm,sibling angle=40,
                    },
                    edge from parent/.style={draw, very thin},nodes=fill,
                    level distance=3em,
                    triangle/.style = {fill=Black!100, regular polygon, regular polygon sides=3 },                    
                    border rotated/.style = {shape border rotate=180}
]
    \node[triangle,
        border rotated,
        minimum size=3em,        
        %label={[yshift=-2em]\circled{S}},                
    ] (sink){}
        child [color=Black!80] foreach \A in {0,1,2,3} { node (n\A){}
            child [color=Black!60] foreach \B in {0,1,2} { node (n\A\B){}
                \if\A1 
                    \if\B2                                                    
                        child [color=Black, draw=none, dotted] foreach \C in {0,1,2} { node (n\A\B\C){} edge from parent[draw=none]}                         
                    \else child [color=Black, draw=none] foreach \C in {0,1,2} { node (n\A\B\C){}}   
                    \fi
                \else child [color=Black, draw=none] foreach \C in {0,1,2} { node (n\A\B\C){}}   
                \fi
            }
        };
    
    % Change color of the "relevant" nodes
    \aftercolorof{Dandelion}{n120}
    
    
    
    % Background
    \begin{scope}[on background layer]  
        \fill[CadetBlue!10] (sink.center) circle[radius=4.2cm];
        
        %% Rime %%
        % South-east
        \fill[PineGreen!50] (n10.center) circle[radius=1em];
        \fill[PineGreen!50] (n11.center) circle[radius=1em];
        \fill[PineGreen!50] (n12.center) circle[radius=1em];
        
        % South-west        
        \fill[PineGreen!50] (n20.center) circle[radius=1em];
        
        % Add connections of yellow's siblings
        \draw[Black!100, thick, dashed] (n120) -- (n12);
        % \draw[Black!100, thick, dotted] (n120) -- (n20);

        \draw[Black!100, very thin] (n121) -- (n12);
        \draw[Black!100, very thin] (n122) -- (n12);


    \end{scope}
    
    % Legend    
    \begin{customlegend}[
        legend columns = 3,
        legend style={
            % the /tikz/ prefix is necessary here...
            % otherwise, it might end-up with `/pgfplots/column 2`
            % which is not what we want. compare pgfmanual.pdf
            /tikz/column 2/.style={column sep=1em},
            /tikz/column 4/.style={column sep=1em}
        },
        legend entries={ % <= in the following there are the entries
            Edge device,            
            Sink,            
            Gateway node,
            Parent candidate,          
            Connection to parent,
            Node searching a parent,   
        },
        legend style={at={(5.25,-4.5)},font=\footnotesize}] % <= to define position and font legend
        % the following are the "images" and numbers in the legend        
        \addlegendimage{shape in legend={circle, yshift=0.2em}{black}}        
        \addlegendimage{shape in legend={triangle, border rotated, inner sep=1.2, yshift=0.3em}{black}}        
        \addlegendimage{shape in legend={diamond, yshift=0.2em}{black}}
        \addlegendimage{shape in legend={draw=none, yshift=0.1em}{PineGreen!60}}        
        \addlegendimage{Black, dashed, very thick}
        \addlegendimage{shape in legend={circle, draw=none, yshift=0.1em}{Dandelion}}
    \end{customlegend}      
\end{tikzpicture}
\end{document}