%!TEX root = ../main.tex
\section{Future Work}
\label{sec:future-work}
\steffen{sorry this is way too much for future work and future work comes after the conclusion}
\sout{
In this section we cover some open points that we would like to address in the future with Rime.
First, we would like to further quantify \textit{i)} the relationship between query selectivity and 
messaging overhead of Rime compared to a naive query dissemination approach and \textit{ii)} 
the frequency of movement updates where using query dissemination becomes 
inefficient and automatically switch between execution models.
\steffen{so now you shoot yourself in your own foot by pointing the reviewer to all shortcomings of you approach, please remove this}

Second, we do not address the questions of \textit{i) how} Rime can help with 
query optimization and \textit{ii) how} to improve relational-operator placement.
By combining \textit{SRTs} over multiple attributes, Rime can potentially increase the performance
of queries with a selection operation in a \textit{WHERE} clause. This is interesting
since currently Rime only supports one \textit{AT} over a single attribute.

Finally, we envision Rime as an extension to 
load balancing and efficient query processing in SPEs. Regarding load balancing, 
Rime helps identify neighboring nodes to offload processing or perform
load shedding. Since we focus on organizational aspects of the IoT and on efficient query 
dissemination, an SPE that uses Rime can use different routing paths for results 
routing (sources to sink) than for query routing (sink to sources, with Rime). 
In contrast to query dissemination, the routing of results requires the fastest 
connections \textit{to}, and the highest throughput \textit{at} the downstream nodes. 
Therefore, distinct routing paths for queries and results can potentially improve overall 
performance. In this regard, Rime acts as an overlay to existing 
networks and can be a drop-in extension module for existing SPEs.}

% Finally, current state-of-the-art on emulating IoT testbeds moves
% towards hybrid application deployments. For this paper we used
% Containernet, which is based on Mininet, in order to create a manual
% topology and insert latency depending on the level of the topology.
% In order to further ease the modeling, deployment, and reproducibility of our
% experiments, we consider moving to IoT-first network emulators, such as Fogify.